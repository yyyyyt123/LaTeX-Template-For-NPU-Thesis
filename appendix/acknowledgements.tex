\renewcommand{\baselinestretch}{1.5}
\fontsize{12pt}{13pt}\selectfont
\phantomsection
\chapter*{致~~~~谢}
\addcontentsline{toc}{chapter}{\fHei 致谢}


在完成本篇论文的过程中,我要首先感谢学校组织的海外毕设项目,让我有幸前往香港中文大学,加入徐宏教授的研究组,参与了为期7个月的海外毕设项目。这个宝贵的机会让我能够在国际化的研究环境中学习和成长,拓宽了我的学术视野。

此外,我还要感谢计算机学院对我的支持和培养。特别要感谢学院设置的康继昌智能系统班,作为第一届康班的成员,我有幸结识了很多优秀的同学,我们一起努力学习和进步。在康班的学习生活中,我找到了自己真正热爱的系统网络方向,并在老师的指导下开展了一系列具有挑战性的竞赛和科研项目,收获了宝贵的经验和成长。

其次,我要衷心感谢我的毕设导师崔禾磊教授和陈亚兴教授,以及外校指导老师徐宏教授。他们在整个研究过程中给予了我专业的指导和深入的思路启发。他们的悉心指导使我能够深入理解研究领域的核心问题,并提供了宝贵的建议和意见,使我能够顺利地完成研究工作。

此外,我要感谢我的朋友们对我无私的支持和帮助。特别要感谢我的好朋友张博日常与我讨论,让我有了清晰的研究目标。还有我的舍友张瀚文、张利军和戴柯迪,我们一起度过了在云天苑306A的宝贵时光。他们在整个研究过程中给予了我宝贵的建议和鼓励。我们进行了许多深入的讨论和思考,共同推动了我的研究工作的进展。感谢当时龙芯杯的队友魏天昊,江嘉熙,申世东,与你们一起参加龙芯杯,一起通宵的经历是我大学以来最难忘的回忆。还有康班的其他同学们,向你们学到了很多宝贵的技术知识。

我还要特别感谢我的师兄谭昕和师姐李嘉敏。他们在我的研究方向上提供了许多有益的讨论和指导,帮助我克服了许多困难和挑战。他们的经验和见解对我的研究工作有着重要的影响,使我能够更加深入地探索问题并取得进展。

最后,我要衷心感谢所有帮助和指导过我的人。没有你们的支持和鼓励,我无法完成这篇论文。你们的知识和经验对我产生了深远的影响,激励着我不断学习和进步。

再次向所有帮助过我的人表示衷心的感谢!我将倍加珍惜这段宝贵的学术经历,并将继续努力追求知识的深度和广度,发扬康继昌先生的奉献精神,为解决卡脖子问题做出自己的贡献。

\clearpage
\endinput